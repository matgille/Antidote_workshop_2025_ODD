\documentclass[11pt,aspectratio=1610]{beamer}
\usepackage{fbb}
\usetheme{CambridgeUS}
\beamertemplatenavigationsymbolsempty
\usepackage[utf8]{inputenc}
\usepackage{amsmath,adjustbox,mathtools}
\usepackage{amsfonts}
\usepackage{hyperref}
\usepackage{graphicx}
\usepackage{xpatch}
\usepackage{makecell}
\usepackage{tabularx}
\usepackage{csquotes}
\usepackage{adjustbox}
\setbeamertemplate{caption}{\raggedright\insertcaption\par}
\usepackage{minted}
\usetheme{default}
\usecolortheme{default}


% Définition des couleurs
\definecolor{schTag}{RGB}{204, 0, 0}       % Rouge pour les balises
\definecolor{schAttr}{RGB}{0, 0, 153}      % Bleu pour les attributs
\definecolor{schValue}{RGB}{0, 102, 0}     % Vert pour les valeurs
\definecolor{schBg}{RGB}{245, 245, 245}    % Fond gris clair

% Configuration de minted
\setminted{
  bgcolor=schBg,
  frame=single,
  framesep=2mm,
  linenos=true,
  breaklines=true,
  fontsize=\small,
}


\usepackage{pgfplots}
\usepackage{tikz}
\usetikzlibrary{shapes,calc,matrix,decorations.markings,decorations.pathreplacing,positioning, intersections,backgrounds,through,hobby}
\usepackage[english]{babel}
\AtBeginSection[]
{\begin{frame}
 \frametitle{}  
\setlength{\itemsep}{.5cm}
 \tableofcontents[currentsection,
                  hideothersubsections,
		  subsectionstyle=hide/hide/hide,
                  subsubsectionstyle=show/show/show/hide
                   ]
 \end{frame} 
 }

\AtBeginSubsection[]
{\begin{frame}
 \frametitle{}  
 \tableofcontents[currentsubsection,
	 	  currentsection,
		  sectionstyle=show/hide,
                  subsectionstyle=show/shaded/hide
                   ]
 \end{frame} 
 }



\setbeamertemplate{sections/subsections in toc}[square]
\setbeamertemplate{itemize item}[square]
\setbeamertemplate{enumerate item}[square]
\setbeamertemplate{itemize subitem}[square]

\usepackage[style=verbose,citestyle=authoryear,isbn=true,url=false,
doi=true,backend=biber,maxbibnames=9, maxcitenames=4]{biblatex}
\addbibresource{/home/mgl/Bureau/Travail/Communications_et_articles/bibliographie_commune/biblio.bib}
\renewcommand*{\bibfont}{\tiny} 

\usepackage{tikz}
\usetikzlibrary{shapes,calc,matrix,decorations.markings,decorations.pathreplacing,positioning, intersections,backgrounds}
\tikzstyle{bag} = [align=center]

\date[]{}
\author[Matthias \textsc{Gille Levenson}]{\\~\\ Matthias \textsc{Gille Levenson}\\   {\scriptsize Université Versailles Saint Quentin (UVSQ) \& École Normale Supérieure de Lyon, France}\\ {\tiny matthias [dot] gille [-] levenson [at] ens-lyon [dot] fr}\vspace{-1cm}}
\title[TEI schemata]{Introduction to TEI schemata\\ {\small Antidote workshop, Klosterneuburg \& Lyon}}
%\titlegraphic{\hspace{0.5cm}\includegraphics[scale=0.21]{/home/mgl/Bureau/Travail/admin/logos/ensl.png}\hspace{0.5cm}\includegraphics[scale=0.23]{/home/mgl/Bureau/Travail/admin/logos/ciham.pdf}\hspace{0.5cm}}
\usepackage[justification=centering]{caption}
\begin{document}
\maketitle

\section{Introduction}


\begin{frame}{Where to find these slides}
\begin{itemize}
\item \href{https://github.com/matgille/Antidote_workshop_2025_ODD.git}{https://github.com/matgille/Antidote\_workshop\_2025\_ODD.git}
\end{itemize}
\end{frame}
\subsection{The XML format}


\begin{frame}{Conformance and validity}
\begin{itemize}
\item XML stands for \textbf{eXtensible Markup Language}.
\item It is a format that allows to describe any kind of textual (or numeric) data 
\item It is the actual \textit{format} the TEI uses, but it might change/evolve in the future.
\end{itemize}
\end{frame}

\begin{frame}{Two important concepts: Conformance and validity}
\begin{itemize}
\item A document \textbf{must} be XML conformant, that is, respect the rules of the XML format
\item A document \textbf{may} be validated against a given \textbf{schema}, that is a document that checks given rules are respected
\item Some examples of specifications: TEI, EAD, DublinCore, AltoXML, PageXML, RDF, etc...
\item Along with validity comes another important concept/nightmare: \textit{namespaces}. You'll understand soon enough.
\end{itemize}
\end{frame}


\begin{frame}{Conformance}
\begin{itemize}
\item XML is composed of elements, attributes, attribute values and text.
\end{itemize}
\begin{center}
\begin{figure}
\includegraphics[height=.5\textheight]{img/element_attribute_text.png}
\caption{This is \textit{correct} XML (but incorrect/incomplete TEI)}
\end{figure}
\end{center}
%\begin{itemize}
%\item There is little assumption about what can be  in XML (hence eXtensible).
%\end{itemize}
\end{frame}



\begin{frame}{Recap: vocabulary}
Four main objects (called \textit{nodes}):
\begin{itemize}
\item \textbf{Element} nodes: the basic object of any XML document: \\ \texttt{<elementName/>} or \texttt{<elementName>...</elementName>}
\item \textbf{Attribute} nodes: part of an element that helps specifying it: \\ \texttt{attributeName="value"}
\item \textbf{Text nodes}: every textual object inside an element
\item \textbf{Comment} nodes: content not parsed by the machine: \\ \texttt{<!-{}- some useful comment -{}->}
\end{itemize}
\end{frame}


\begin{frame}{The root node/element}
\begin{itemize}
\item One and only one top element that contains everything else
\end{itemize}
\begin{center}
\begin{figure}
\includegraphics[height=.5\textheight]{img/element_attribute_text.png}
\caption{This is \textit{correct} XML (but incorrect/incomplete TEI)}
\end{figure}
\end{center}
\end{frame}


\begin{frame}{Conformance}
\begin{itemize}
\item \textit{No} overlapping elements
\end{itemize}
\begin{center}
\begin{figure}
\includegraphics[width=.9\textwidth]{img/overlapping.png}
\caption{This is \textit{incorrect} XML}
\end{figure}
\end{center}
\end{frame}


\begin{frame}{Conformance}
\begin{itemize}
\item Some elements can contain other elements, but elements can also be empty
\end{itemize}
\begin{center}
\begin{figure}
\includegraphics[width=.9\textwidth]{img/empty.png}
\caption{This is \textit{correct} XML}
\end{figure}
\end{center}
\end{frame}



\begin{frame}{Conformance}
\begin{itemize}
\item Attribute values must be inside quotes
\end{itemize}
\begin{center}
\begin{figure}
\includegraphics[height=.5\textheight]{img/attributes.png}
\caption{This is \textit{incorrect} XML}
\end{figure}
\end{center}
\end{frame}



\begin{frame}{Conformant or not ?}
\begin{center}
\begin{figure}
\includegraphics[height=.8\textheight]{img/conformant.png}
\end{figure}
\end{center}
\end{frame}



\subsection{What is the TEI?}
\begin{frame}{What is the TEI?}
\begin{itemize}
\item A community 
\item A \enquote{standard} and a set of rules, the Guidelines: \url{https://tei-c.org/release/doc/tei-p5-doc/en/html/}
\item A way of seeing/modelling \enquote{the} text
\end{itemize}
\end{frame}


\subsection{Principles}
\begin{frame}{}
\begin{itemize}
\item Separate the appearance and the \enquote{essence} of textual objects
\end{itemize}
\begin{center}
\begin{figure}
\includegraphics[width=.8\textwidth]{img/texte.png}
\includegraphics[width=.8\textwidth]{img/apparat.png}
\caption{Fragment of an edition of the \textit{Castigos de Sancho IV} \parencite{sanchez_VersionInterpoladaCastigos_2003}}
\end{figure}
\end{center}
\end{frame}

\begin{frame}{}
\begin{itemize}
\item Separate the appearance and the \enquote{essence} of textual objects
\end{itemize}
\begin{center}
\begin{figure}
\includegraphics[width=.8\textwidth]{img/xml_structured.png}
\caption{Its possible representation in XML-TEI}
\end{figure}
\end{center}
\end{frame}


\subsection{The TEI, what for ?}
\begin{frame}{The TEI, what for ?}
\begin{itemize}
\item Describing a text using the experience of a \textbf{large community}
\item Producing semantic data that can be \textbf{read} by the human and \textbf{processed} by the computer
\item Make sharing and reusability of these representations easier
\end{itemize}
\end{frame}




\section{Schemata}
\subsection{Interests of using schemata}
\begin{frame}{What is the use of a schema?}
\begin{itemize}
\item A schema checks if the data is correctly formatted.
\item A schema makes assumptions about what is legal and what is illegal in an XML data structure
\item Applied to some text encoding, a schema is a model of the \enquote{text}, that is, a \textbf{formal representation} of the text.
\item From a pragmatic point of view, it helps harmonizing your encoding and is an \textbf{important step towards interoperability}
\end{itemize}
\end{frame}


\begin{frame}{Integrating the schema in your encoding sessions}
Some XML editors use the schema as a tool for easing the encoding: 
\begin{figure}
 \includegraphics[height=.5\textheight]{img/example_tei.png}
 \caption{Oxygen XML Editor allows you to choose between the allowed elements}
\end{figure}
\end{frame}

\begin{frame}{Schemas}
\begin{itemize}
\item A schema is a document that is used to control the quality of some encoding.
\item The TEI provides \textbf{rules} (the \textit{Guidelines}, human readable), and their \textbf{formalization} (\textit{schemas}, machine readable) to check the validity of a given document
\item From an abstract point of view, the schema represents the formalization of \textit{your} modelling of a given text or genre.
\end{itemize}
\end{frame}


\begin{frame}{Schemas}
\begin{center}
\begin{figure}
\includegraphics[height=.75\textheight]{img/validation.png}
\caption{This fragment is \textit{well-formed}, but it is not \textit{valid} according to the XML-TEI schema. A paragraph \texttt{p} or an anonymous block \texttt{ab} should wrap the lines.}
\end{figure}
\end{center}
\end{frame}


\begin{frame}{Schema formats}

There are multiple schema formats. The most important are:
\begin{itemize}
\item Document Type Definition (DTD), already mentionned before
\item XML schema (XSD format)
\item Relax NG (RNG) schemas and its compact version (RNC)
\end{itemize}
The former is less used nowadays but can prove usefull for creating shortcut elements thanks to \textbf{XML entities} (e.g. for abbreviation encoding)

\begin{center}
\begin{figure}
\includegraphics[height=.3\textheight, trim={0 1cm 0 2cm}, clip]{img/entities.png}
\caption{A possible way to encode abbreviations using XML entities / DTD definitions}
\end{figure}
\end{center}
\end{frame}


\subsection{Schemas and the TEI I}
\begin{frame}{TEI subsets}
\begin{center}
\begin{figure}
\includegraphics[height=.6\textheight]{img/schemas.png}
\end{figure}
\begin{itemize}
\item There are multiple TEI schemas available.
\item As a scientific standard, the TEI needs to be adapted to the user’s needs.
\end{itemize}
\end{center}
\end{frame}


\begin{frame}{Why should you create your own TEI customization ?}
\begin{center}
\begin{itemize}
\item Consider the TEI as a tool to model your textual object
\item You probably won't need all the intellectual tools provided by the TEI Consortium.
\item As medievalists, you might want to discard the elements belonging to the \enquote{Transcriptions of Speech} section of the TEI (alas!)
\item Your schema will be the formalization of your way of seeing the text, and you need it to be as specific as possible.
\end{itemize}
\end{center}
\end{frame}


\begin{frame}{Two main methods}
\begin{center}
\begin{itemize}
\item The inclusion method: start from scratch, add the elements you need
\item The exclusion method: start from a predefined schema, remove the elements you won't need
\pause\item  I would recommended to adopt the first approach...
\pause\item or at least starting from a preset specification that meets some of your interests (we'll see this later).
\pause\item Mastering the Guidelines is essential to know what is possible for your project.
\end{itemize}
\end{center}

\end{frame}



\begin{frame}{Don't reinvent the wheel !}
\begin{center}
\begin{figure}
\includegraphics[width=.4\textwidth]{img/standards.png}
\caption{\url{https://xkcd.com/927/}}
\end{figure}
\begin{itemize}
\item You have to specify, and adapt the TEI rules to your own project
\item However, the modelling problems you will face are probably \textit{not} new
\item Please don't reinvent the wheel ! 
\item Search for what's been done for similar problems and try the provided solutions !
\item And (for a start), avoid inventing/creating new elements.
\end{itemize}
\end{center}
\end{frame}


\subsection{Schemas and the TEI II: ODD's}
\begin{frame}{Is that all, folks ?}
\begin{center}
\begin{itemize}
\item Create a specific schema is a good start.
\pause\item What's missing then ? \pause \textit{Documentation}.
\pause\item You want now to explain your rules to your supervisor, to your colleagues and to your children. You need to write this \textit{in natural language}.
\pause\item What could be better than a TEI document to describe another TEI document?
\end{itemize}
\end{center}
\end{frame}








\begin{frame}{Here comes the ODD}
\begin{itemize}
\item ODD stands for \enquote{One Document Does it All}
\item An ODD is a TEI document that is used to create both the documentation and the schema in a single document.
\item Important reading: \fullcite{burnard_WhatTEIConformance_2019}
\end{itemize}
\end{frame}


\begin{frame}{One Document to Rule Them All}
An ODD contains:
\begin{itemize}
\item the machine-readable elements used to \textit{validate} an edition
\item the human-readable elements used to \textit{document} it
\end{itemize}
\end{frame}






%\begin{frame}{Basic TEI schema concepts \parencite{vigliantiOneDocumentDoesitall2019a}}
%\begin{itemize}
%\item \textbf{Module}
%\item \textbf{Model class} (or element class)
%\item Attribute class
%\item Element
%\item Classes
%\item datatypes
%\item macros
%\end{itemize}
%You will manipulate the items in bold for a start. 
%\end{frame}


\begin{frame}
       \begin{columns}
          \column{0.55\linewidth}
\begin{figure}
             \centering
 \includegraphics[width=.6\textwidth]{img/tei_modules.png}
\end{figure}
           \column{0.40\linewidth}
\begin{itemize}
\item The TEI is organized by \textit{thematic} modules
\end{itemize}
         \end{columns} 
    \end{frame}


\begin{frame}{TEI as modules}
\begin{itemize}
\item 22 modules
\item Each module is binded to a list of elements. Deactivate a module and those elements won't be accessible anymore. 
\item 4 main modules: \texttt{tei}, \texttt{core}, \texttt{header}, and \texttt{textstructure}, almost mandatory
\item The other module you might want to take a look at are: \texttt{msdescription}, (\texttt{transcr}) and \texttt{textcrit}
\end{itemize}
\end{frame}



\begin{frame}{An \textsc{Odd} is NOT a schema}

\begin{itemize}
\item It is a \textbf{TEI document} used to produce schemata.
\item It needs to be \textbf{converted} to a schema (and a documentation when applicable)
\item The resulting schema (in general, in RNG) is the document you will use to validate your XML documents.
\item Don't try to validate your files against ODD's !
\end{itemize}
\end{frame}





\subsection{ROMA}
\begin{frame}{Producing ODD's and schemas with ROMA}

% Image vers ROMA
\begin{itemize}
\item ROMA, maintained by Raffaele Viglianti (lead), Peter Stadler and Bryan Wang.
\item ROMA is a simple interface for designing ODDs and creating schemas
\item URL: \url{https://roma.tei-c.org/}
\end{itemize}
\end{frame}


\begin{frame}{A quick look at our ODD}
With ROMA, you can easily design your own ODD:
\begin{itemize}
\item Include or exclude specific modules
\item Completely remove elements (they won't be allowed anymore)
%\item Partially remove elements (they will be allowed in specific contexts only)
\item Constrain attribute types, or attribute values 
\item Make some attributes mandatory
\item etc.
\end{itemize}
\end{frame}


\begin{frame}{Producing ODD's and schemas with ROMA}

% Image vers ROMA
\begin{itemize}
\item I'm interested in the encoding of a medieval manuscript. How can I create the most adapted ODD with ROMA?
\item I want to make sur the user uses an attribute \texttt{@n} for each page beginning element (\texttt{pb}). How can I do ?
\item I want to remove the \texttt{div1}, \texttt{div2}, ..., \texttt{div7} elements, so that divisions are marked up with typed and identified \texttt{div} elements only. 
\end{itemize}
\end{frame}

\begin{frame}{Producing ODD's and schemas with ROMA}
\begin{figure}
 \includegraphics[width=.9\textwidth]{img/roma_main_page.png}
 \caption{ROMA main page: presets}
\end{figure}
\end{frame}

\begin{frame}{Producing ODD's and schemas with ROMA}
\begin{figure}
 \includegraphics[width=.9\textwidth]{img/settings_odd.png}
 \caption{Configuring your ODD}
\end{figure}
\end{frame}

\begin{frame}{Producing ODD's and schemas with ROMA}
\begin{figure}
 \includegraphics[width=.9\textwidth]{img/finding_pb.png}
 \caption{Find the element you want to modify (here, \texttt{pb})}
\end{figure}
\end{frame}

\begin{frame}{Producing ODD's and schemas with ROMA}
\begin{figure}
 \includegraphics[width=.9\textwidth]{img/feature_selection.png}
 \caption{Choose the feature you want to adapt (documentation or attribute)}
\end{figure}
\end{frame}

\begin{frame}{Producing ODD's and schemas with ROMA}
\begin{figure}
 \includegraphics[width=.9\textwidth]{img/attribute_selection.png}
 \caption{Find the attribute you want to modify}
\end{figure}
\end{frame}

\begin{frame}{Producing ODD's and schemas with ROMA}
\begin{figure}
 \includegraphics[width=.9\textwidth]{img/changing_attribute_behaviour.png}
 \caption{Change its usage ! It's (almost) automagical !}
\end{figure}
\end{frame}


\begin{frame}{The produced ODD}
\begin{figure}
 \includegraphics[width=.65\textwidth]{img/my_first_ODD.png}
\end{figure}
I find this specification too broad. I would remove some modules. Do you want to describe figures? Are you interested in named entities?
\end{frame}



\begin{frame}{Producing ODD's and schemas with ROMA: exporting the ODD}

% Image vers ROMA
\begin{itemize}
\item Multiple output formats. 
\item The most important is ODD for scientific purposes.
\item You'll need RNC/RNG for validation purposes. 
\item You won't be able to retrieve your ODD from an RNG or any other schema file.
\item Oxygen can also produce the schema from the ODD's. 
\end{itemize}
\end{frame}



\begin{frame}{Making a reference to the schema in your TEI file}
\begin{figure}
 \includegraphics[width=1\textwidth, trim={0 0 0 .2cm}, clip]{img/linking.png}
 \caption{Before the root node, the reference to a DTD (in blue, second line) and to my ODD derived schema (in violet, third line).}
\end{figure}
\end{frame}




\begin{frame}{A quick look at our ODD}
Let's sum up. You can:
\begin{itemize}
\item Completely remove elements (they won't be allowed anymore)
\item Partially remove elements (they will be allowed in specific contexts only)
\item Constrain attribute types, or attribute values; make some attributes mandatory
\end{itemize}
How does it work ? 
\end{frame}

\begin{frame}[fragile]{Activate a module and include selected elements}
\begin{minted}{xml}
<moduleRef key="textcrit" include="listWit witness witEnd witStart app rdg"/>
\end{minted}
\begin{itemize}
\item The \texttt{include} attribute means a rebuilding of the module from scratch: absent elements won't be allowed in your encoding
\item Note here the choice not to include the \texttt{lemma} element, derived from a specific editorial methodology (\enquote{variance}, etc). 
\end{itemize}
\end{frame}

\begin{frame}[fragile]{Make attributes mandatory}
\begin{minted}{xml}
<elementSpec ident="pb" mode="change">
  <attList>
     <attDef ident="n" mode="change" usage="req"/>
  </attList>
</elementSpec>
\end{minted}
\end{frame}

\begin{frame}[fragile]{Remove attributes at element level}
\begin{minted}{xml}
<elementSpec ident="material" mode="change" module="msdecription">
    <attList>
        <attDef ident="function" mode="delete"/>
        <attDef ident="target" mode="delete"/>
        <attDef ident="resp" mode="delete"/>
        <attDef ident="ref" mode="delete"/>
    </attList>
</elementSpec>
\end{minted}
\end{frame}

\begin{frame}[fragile]{Constrain attribute values}
\begin{minted}[fontsize={\fontsize{9.5}{10.5}\selectfont}]{xml}
<elementSpec ident="material" mode="change" module="msdecription">
    <attList>
        <attDef ident="type" mode="replace" usage="req">
            <valList mode="add" type="closed">
                <valItem ident="parchment">
                    <desc>The copy medium is parchment.</desc>
                </valItem>
                <valItem ident="paper">
                    <desc>The copy medium is paper.</desc>
                </valItem>
                <valItem ident="mixt">
                    <desc>The copy medium alternates between paper and parchment.</desc>
                </valItem>
            </valList>
        </attDef>
    </attList>
</elementSpec>
\end{minted}
\end{frame}



\begin{frame}{The other side of ODD's: natural language documentation}
% Image vers ROMA
\begin{itemize}
\item Roma is just the first step to create the machine-readable version of your ODD
\item As said before, you'll need to create and write the documentation \textit{in natural language}, using the TEI. Example: \href{https://gitlab.huma-num.fr/mgillelevenson/hyperregimiento-de-los-principes/-/blob/master/Dedans/XML/schemas/ODD_transcriptions.xml?ref_type=heads}{here}.
\item You can use any XML editor to do so.
\end{itemize}
\begin{figure}
\includegraphics[height=.20\textheight]{img/example_ODD.png}
\caption{Documenting the use of \texttt{milestone} in the edition of the \textit{Regimiento de los prínçipes}.}
\end{figure}
\end{frame}


\begin{frame}{An example of complete ODD}
\begin{itemize}
\item An ODD is an XML-TEI document divided in two parts: the documentation and the schema.
\item For structuring, use \texttt{div1}...\texttt{div9} elements
\item You can encode your documentation as you would encode any document
\item And put you schema specification (the \texttt{schemaSpec}) in a \texttt{div1} for instance.  
\end{itemize}
\end{frame}

\begin{frame}{An example of complete ODD}
% Image vers ROMA
\begin{figure}
\includegraphics[height=.80\textheight]{img/final_ODD.png}
\caption{Documenting the use of \texttt{milestone} in the edition of the \textit{Regimiento de los prínçipes}.}
\end{figure}
\end{frame}


%\begin{frame}{Respect datatypes plz}
%% Image vers ROMA
%\begin{itemize}
%\item Please make sure you use correctly the elements and attributes described by the TEI consortium
%\item In particular, be aware of datatypes:
%\item For instance, link attributes of class \texttt{teidata.pointer} (\texttt{@ana, @corresp, @sameAs}, etc) are \textit{pointers}, i.e., their datatype is an URI. In general, the URI used in the TEI are relative links, 
%\item When they are not URL, they \textbf{must} start with the hashtag \texttt{\#}.
%\end{itemize}
%\end{frame}



\begin{frame}[fragile]{Fine-tune the validation with Schematron}
\begin{minted}{xml}
<!--The following element can be a child of any elementSpec node-->
<constraintSpec xmlns:sch="http://purl.oclc.org/dsdl/schematron" scheme="isoschematron">
    <constraint>
      <sch:rule context="pb">
         <sch:assert
            test="matches(@n, '\d+[vr]')"
            >Something is wrong with the foliation of this page. Have you forgotten the page side indication ?</sch:assert>
      </sch:rule>
    </constraint>
</constraintSpec>
\end{minted}
\end{frame}


\section{Conclusions}
\begin{frame}{Conclusions}
\begin{itemize}
\item Schemas even minimal, are essential for any scientific publishing project based on TEI.
\item TEI-based editions without the publication of XML sources is quite unuseful and won't last much
\item Please provide access to your data (and its documentation) !
\end{itemize}
\end{frame}


\section{Exercise}
\begin{frame}{Exercises}
\begin{itemize}
\item The exercise instructions are in the xml file \texttt{bentham.xml}
\end{itemize}
\end{frame}


\section{One more thing: ODDbyexample}
\begin{frame}{ODDbyexample}
\begin{itemize}
\item ODDByExample is a tool to produce an ODD using an already existing XML-TEI document
\item It will create the ODD that matches this document
\item Very useful tool ...
\item ... that can be tricky if the XML-TEI example shows inconsistency: they will be embedded into your specification
\item Make sure to use this tool with small, carefully checked fragments of your document.
\item Link: \href{https://github.com/TEIC/Stylesheets/blob/dev/tools/oddbyexample.xsl}{https://github.com/TEIC/Stylesheets/blob/dev/tools/oddbyexample.xsl}
\end{itemize}
\end{frame}

\begin{frame}{References}
\printbibliography
\end{frame}

\end{document}
